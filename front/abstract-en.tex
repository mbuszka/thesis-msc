\englishabstract{%
This thesis presents a robust algorithm for automatically deriving an abstract machine corresponding to a given interpreter.

The transformation is based on a manual derivation technique known as the functional correspondence between evaluators and abstract machines which consists of two source code transformations: translation to continuation-passing style which uncovers the control flow of the interpreter  and defunctionalization which produces a first-order program.

The algorithm begins with the translation to administrative normal form which eases program analysis and the subsequent steps of the transformation: selective translation to continuation-passing style and selective defunctionalization.
Both transformations extend their classical formulations with the ability to transform only the desired parts of the program by utilizing the control-flow analysis which is computed by an abstract interpreter obtained using the abstracting abstract machines methodology.

The thesis is accompanied by an implementation of the algorithm in the form of a command-line tool.
It allows for automatic transformation of an interpreter embedded in a \textit{Racket} source file and gives fine-grained control over the resulting machine.
A selection of case-studies is presented to showcase the performance of the tool and the algorithm by deriving both known and novel abstract machines.
}