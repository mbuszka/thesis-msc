\polishabstract{%
W pracy przedstawiony jest algorytm pozwalający na automatyczną derywację maszyny abstrakcyjnej odpowiadającej źródłowemu interpreterowi.

Transformacja oparta jest na ręcznej metodzie derywacji znanej jako odpowiedniość funkcyjna pomiędzy ewaluatorami i maszynami abstrakcyjnymi, która składa się z dwóch transformacji kodu źródłowego: przekształcenia do stylu przekazywania kontynuacji ujawniającego przepływ sterowania w interpreterze oraz defunkcjonalizacji, która pozwala na uzyskanie pierwszorzędowego programu.

Algorytm rozpoczyna się translacją do administracyjnej postaci normalnej która ułatwia analizę programu oraz dalsze przekształcenia: wybiórczą transformację do stylu przekazywania kontynuacji oraz wybiórczą defunkcjonalizację.
Obie transformacje są rozszerzeniem klasycznych sformułowań i umożliwiają przekształcanie wybranych części programu dzięki zastosowaniu analizy przepływu sterowania, która jest obliczana za pomocą abstrakcyjnego interpretera otrzymanego z wykorzystaniem metodologii abstrahowania maszyn abstrakcyjnych.

Do pracy dołączona jest implementacja algorytmu w postaci programu używanego z wiersza poleceń.
Pozwala ona na automatyczną transformację interpretera zanużonego w pliku źródłowym w języku \textit{Racket}, jednocześnie zapewniając precyzyjną kontrolę nad kształtem wynikowej maszyny.
W pracy przedstawiono zbiór przykładowych interpreterów na których zobrazowano działanie narzędzia i algorytmu poprzez derywację zarówno znanych jak i nowych maszyn abstrakcyjnych.
}