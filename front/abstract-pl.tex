\polishabstract{%
W pracy pokazuję jak przekształcić ręczną metodę derywacji, znaną jako odpowiedniość funkcyjna pomiędzy ewaluatorami i maszynami abstrakcyjnymi, w uniwersalny algorytm.

Zaczynając od klasycznego sformułowania metodologii, składającego się z dwóch transformacji kodu źródłowego wymagających pracy człowieka, wskazuję na analizę przepływu sterowania jako podstawę do ich algorytmizacji.
Następnie definiuję meta-język \IDL{} i przedstawiam trzy główne etapy automatycznej procedury: transformację do adminstracyjnej postaci normalnej, wybiórczą transformację do stylu przekazywania kontynuacji oraz wybiórczą defunkcjonalizację.
Pokazuję także procedurę pozwalającą na obliczenie analizy przepływu sterowania dla programów w \IDL{}, którą otrzymałem stosując metodologię abstrahowania maszyn abstrakcyjnych.

Do pracy dołączona jest implementacja algorytmu w postaci programu używanego z wiersza poleceń.
Pozwala ona na automatyczną transformację interpretera zanużonego w pliku źródłowym w języku \textit{Racket}, jednocześnie zapewniając precyzyjną kontrolę nad kształtem wynikowej maszyny.
W pracy przedstawiam zbiór przykładowych interpreterów na których obrazuję działanie narzędzia i algorytmu poprzez derywację zarówno znanych jak i nowych maszyn abstrakcyjnych.
}