\documentclass[handout]{beamer}

\mode<presentation>
{
  \usetheme{Boadilla}      % or try Darmstadt, Madrid, Warsaw, ...
  \usecolortheme{default} % or try albatross, beaver, crane, ...
  \usefonttheme{default}  % or try serif, structurebold, ...
  \setbeamertemplate{navigation symbols}{}
  \setbeamertemplate{caption}[numbered]
} 
\usepackage[polish]{babel}
\usepackage{amsmath}
\usepackage{amssymb}
\usepackage[T1]{fontenc}
\usepackage{multirow}
\usepackage{makecell}
\usepackage{amsfonts}
\usepackage{stmaryrd}


\usepackage{xcolor}
\usepackage{listings}
\usepackage[scale=0.8]{FiraMono} 

\lstdefinelanguage{idl}{%
  alsoletter={-?\#:},
  otherkeywords={(,),\{,\},[,]},
  morekeywords={def,fun,def-data,def-struct,match,let,error},
  morekeywords=[2]{String,Integer,Any,Boolean},
  morekeywords=[3]{(,),\{,\},[,]},
  morekeywords=[4]{\#:bar,\#:apply,\#:atomic,\#:no-defun,\#:name,\#t, \#f, \#:foo},
  sensitive=true,
  morestring=[b]",
  morecomment=[l]{;;}
}

\lstset{%
  basicstyle=\ttfamily\color{black},
  mathescape=true,
  showstringspaces=false,
  keywordstyle={\color{green!50!black}},
  keywordstyle=[2]{\color{blue!50!black}}, 
  keywordstyle=[3]{\color{gray}},
  keywordstyle=[4]{\color{blue!80!black}},
  commentstyle=\color{gray}\itshape,
  stringstyle=\color{green!80!black},
  identifierstyle=\color{black},
  escapechar=@,
  language=idl
}

\expandafter\patchcmd\csname \string\lstinline\endcsname{%
  \leavevmode
  \bgroup
}{%
  \leavevmode
  \ifmmode\hbox\fi
  \bgroup
}{}{%
  \typeout{Patching of \string\lstinline\space failed!}%
}

\newcommand{\Redex}{\texttt{PLT Redex}}
\newcommand{\Racket}{\texttt{Racket}}
\newcommand{\LC}{\(\lambda\)-calculus}

\newcommand{\IDL}{\textit{IDL}}
\newcommand{\bb}[1]{\left\llbracket #1 \right\rrbracket}
\newcommand{\anf}[2]{\bb{#1}#2}
\newcommand{\atomic}[1]{[ #1 ]_{a}}
\newcommand{\anfSeq}[2]{\bb{#1}_{s} #2}
\newcommand{\e}[1]{\lstinline{#1}}
\newcommand{\tuple}[1]{\left< #1 \right>}
\newcommand{\VA}[0]{\mathit{VAddr}}
\newcommand{\KA}[0]{\mathit{KAddr}}
\newcommand{\VS}[0]{\mathit{VStore}}
\newcommand{\KS}[0]{\mathit{KStore}}

\newcommand{\cps}[2]{\bb{#1}_c #2}
\newcommand{\dir}[1]{\bb{#1}_d}
\newcommand{\trivial}{\mathit{trivial}}
\newcommand{\atom}{\mathit{atomic}}
\newcommand{\allAtomic}{\mathit{allAtomic}}
\newcommand{\noneAtomic}{\mathit{noneAtomic}}
\newcommand{\defun}{\mathit{defun}}
\newcommand{\mkBranch}{\mathit{mkBranch}}
\newcommand{\mkApply}{\mathit{mkApply}}
\newcommand{\semt}{\lstinline!semt!}


\title[The Functional Correspondence Applied]{The Functional Correspondence Applied:\\ An Implementation of a Semantics Transformer}
\author[Maciej Buszka]{Maciej Buszka}
\institute[II UWr]{Instytut Informatyki UWr}
\date{18.11.2020}

\begin{document}

\begin{frame}
  \titlepage
\end{frame}

\begin{frame}
  \centering
  \LARGE{The Functional Correspondence}\\
  \tiny{and an introduction to the meta-language}
\end{frame}

% \begin{frame}{Outline}
%   \tableofcontents
% \end{frame}


\section{Introduction}
\begin{frame}{The Functional Correspondence}
  \begin{itemize}
    \item A method for deriving abstract machines\pause
    \item Takes a definitional interpreter
    \item Produces an encoding of an abstract machine\pause
    \item Introduced by Danvy et~al. in \textit{A functional correspondence between evaluators and abstract machines}
    \pause
    \item Based on two source-to-source transformations \pause
    \begin{itemize}
      \item Translation to continuation-passing style (CPS)
      \item Defunctionalization
    \end{itemize}
  \end{itemize}
\end{frame}

\begin{frame}[fragile]
  \frametitle{Meta-language by Example: CBV \LC{}}
  \begin{lstlisting}
(def-data Term
  String            ;; variable
  {Abs String Term} ;; lambda abstraction
  {App Term Term})  ;; application

;; Environment is encoded as a partial function

(def extend (env x v) ... ) ;; extends env at x with value v

(def eval (env term)
  (match term
    ([String x]   (env x))
    ({Abs x body} (fun (v) (eval (extend env x v) body)))
    ({App fn arg} ((eval env fn) (eval env arg)))))
  \end{lstlisting}
\end{frame}

\begin{frame}
  \begin{center}
    \LARGE{Transformation Steps We Aim For}\\
    \tiny{(Also a reminder of how functional correspondence works)}
  \end{center}
\end{frame}

\begin{frame}[fragile]
  \frametitle{After CPS Translation}
  \begin{lstlisting}
(def eval (env term cont)
  (match term
    ([String x] (cont (env x)))
    ({Abs x body}
      (cont (fun (v cont)
        (eval (extend env x v) body cont))))
    ({App fn arg}
      (eval env fn (fun (fn)
        (eval env arg (fun (arg) (fn arg cont))))))))
  \end{lstlisting}
\end{frame}

\begin{frame}[fragile]
  \frametitle{After Defunctionalization}
  \begin{lstlisting}
(def eval (env term cont)
  (match term
    ([String x]   (continue cont (env x)))
    ({Abs x body} (continue cont {Fun body env x}))
    ({App fn arg} (eval env fn {App1 arg cont env}))))

(def apply (fn v cont)
  (match fn ({Fun body env x}
    (eval (extend env x v) body cont))))

(def continue (cont val)
  (match cont
    ({App1 arg cont env} (eval env arg {App2 cont val}))
    ({App2 cont fn}      (apply fn val cont))
    ({Halt}              val)))
  \end{lstlisting}
\end{frame}

\begin{frame}{The Result}
  \begin{itemize}
    \item An abstract machine encoded as mutually tail-recursive functions \pause
    \item Uses abstract environment (encoded as a function)
    \begin{itemize} \pause
      \item \lstinline{extend} is still higer-order
      \item \lstinline{extend} and functions representing environment remain in direct style
    \end{itemize}
  \end{itemize}
\end{frame}

\begin{frame}{Desired Features}
  \begin{itemize}
    \item Metalanguage expressivity \pause
    \begin{itemize}
      \item Mutually recursive, anonymous and higher-order functions \pause
      \item Records and pattern matching \pause
      \item Dynamic (strong) typing \pause
    \end{itemize}\pause
    \item Automation \pause
    \begin{itemize}
      \item Function space detection\pause
      \item CPS and defunctionalization\pause
      \item Sensible name generation
    \end{itemize}\pause
    \item User's control over transformation\pause
    \begin{itemize}
      \item Naming of records for defunctionalized lambdas, apply functions\pause
      \item Disabling defunctionalization and/or cps transformation of chosen functions \pause
    \end{itemize}
  \end{itemize}
\end{frame}

\begin{frame}[fragile]{Why Control is Needed}
  \begin{lstlisting}
(def extend (env y v cont1) (continue cont1 {Ext env v y}))
(def lookup (fn1 x cont2) ... (continue2 ...))

(def eval (env term cont3)
  (match term
    ([String x] (lookup env x cont3))
    ({Abs x body} (continue2 cont3 {Fun body env x}))
    ({App fn arg} (eval env fn {App1 arg cont3 env}))))

(def continue2 (fn4 var3)
  (match fn4
    ({App2 cont3 var2} (apply var2 var3 cont3))
    ({App1 arg cont3 env} (eval env arg {App2 cont3 var3}))
    ({Halt } var3)))

(def apply (fn2 v cont4) ... (extend ...))

(def continue (fn3 var1) ... (eval var1 body cont4))
  \end{lstlisting}
\end{frame}

\begin{frame}
  \begin{center}
    \LARGE{Automating the Functional Correspondence}\\
    \tiny{(A particular approach)}
  \end{center}
\end{frame}

\begin{frame}{In a Nutshell}
  \begin{itemize}
    \item Transformation to ANF \pause
    \item Control flow analysis \pause
    \item Transformation to CPS \pause
    \item Control flow analysis \pause
    \item Defunctionalization \pause
    \item Inlining administrative let-bindings
  \end{itemize}
\end{frame}

\begin{frame}[fragile]{Transformation to ANF}
  \begin{itemize}
    \item Assumes left-to-right call-by-value semantics \pause
    \item Let-bind every intermediate result \pause
    \item Eases program analysis and further transformations \pause
    \item Preserves tail calls
  \end{itemize}\pause
  \vspace{1\baselineskip}
  \begin{beamerboxesrounded}{Abstract syntax of the meta-language}
  \begin{center}
    \begingroup
    \setlength{\tabcolsep}{2pt}
    \def\arraystretch{1.3}
    \begin{tabular}{rrl}
      $x, y, z \in Var$ && $r\in StructName$\\
      $s \in String$ && $b \in Int \cup Boolean \cup String$\\
      $Tp \ni \mathit{tp} $ &::=& \lstinline!String! | \lstinline!Integer! | \lstinline!Boolean!\\
      $Pattern \ni p $ &::=& $x$ | $b$ | \lstinline!_! | \lstinline!{$r$ $p\ldots$}! |  \lstinline![$\mathit{tp}$ $x$]!\\
      $Term \ni t$ &::=& $x$ | $b$
                  | \lstinline!(fun ($x\ldots$) $t$)!
                  | \lstinline!($t$ $t\ldots$)!
                  | \lstinline!{$r$ $t\ldots$}!\\
                  &|& \lstinline!(let $p$ $t$ $t$)!
                  | \lstinline!(match $t$ ($p$ $t$)$\ldots$)!
                  | \lstinline!(error $s$)!
    \end{tabular}
    \endgroup
    \end{center}
  \end{beamerboxesrounded}
\end{frame}

\begin{frame}[fragile]{Transformation to ANF}
  \begin{beamerboxesrounded}{Abstract syntax in ANF}
    \begin{center}
      \begingroup
      \setlength{\tabcolsep}{2pt}
      \begin{tabular}{rll}
        $Com \ni c $ && ::= $x$ | $b$
        | \lstinline!(fun ($x\ldots$) $e$)!
        | \lstinline!($x$ $x\ldots$)!\\
        &&| \lstinline!{$r$ $x\ldots$}!
        | \lstinline!(match $x$ ($p$ $e$)$\ldots$)!\\
        $Anf \ni e $ && ::= $c$ 
        | \lstinline!(let $p$ $c$ $e$)!
        | \lstinline!(error $s$)!
      \end{tabular}\vspace{\baselineskip}
      \endgroup
      \end{center}
  \end{beamerboxesrounded}
  \vspace{\baselineskip}\pause
  \begin{beamerboxesrounded}{ANF translation}
    \begin{center}
      \begingroup
      \setlength{\tabcolsep}{2pt}
      \begin{tabular}{rll}
        $\bb{\cdot}$ &$\cdot$ &: $Term \times (Com \rightarrow Anf) \rightarrow Anf$\\
        $\bb{a}$ &$k$ &$= k\,a$\\
        
        $\bb{\lstinline!(fun ($x\ldots$)\ $t$)!}$ &$k$
        & $= k\, \lstinline!(fun ($x\ldots$) $\anf{t}{id}$)!$\\
        
        $\bb{\lstinline!($t_f \; t_{arg}\ldots$)!}$ &$k$ 
        &$= \anf{t_f}{\atomic{\lambda x_f . \anfSeq{t_{arg}\ldots}{\lambda (x_{arg}\ldots) . k \,\lstinline!($x_f\;x_{arg}\ldots$)!}}}$\\
      
        $\bb{\lstinline!(let\ $x\;t_1\;t_2$)!}$ & $k$
        &$= \anf{t_1}{\lambda c_1 . \lstinline!(let\ $x\;c_1\;\anf{t_2}{k}$)!}$\\
      
        $\bb{\lstinline!\{$r \; t\ldots$\}!}$ &$k$ 
        &$= \anfSeq{t\ldots}{\lambda (x\ldots) . k \,\lstinline!\{$r\;x\ldots$\}!}$\\
      
        $\bb{\lstinline!(match\ $t\;$($p \;t_b$))!}$ & $k$
        &$= \anf{t}{\atomic{\lambda x . k\,\lstinline!(match\ $x\;$($p\;\anf{t_b}{id}$)!}}$\\
      
        $\bb{\lstinline!(error\ $s$)!}$ & \_ & $= $ \lstinline!(error $s$)!
      \end{tabular}
      \endgroup
      \end{center}
  \end{beamerboxesrounded}
\end{frame}

\begin{frame}[fragile]{Transformation to ANF}
  \begin{beamerboxesrounded}{Extending atomic continuation}
    \begin{center}
      \begingroup
      \setlength{\tabcolsep}{2pt}
      \begin{tabular}{rll}
        $[\cdot]_a$ & $\cdot$ & : $(Var \rightarrow Anf) \rightarrow Com \rightarrow Anf$\\
        $[k]_a$ & $x$ & $= k\,x$\\
        $[k]_a$ & $c$ & $= $ \lstinline!(let $x$ $c$ $(k\,x)$)!
      \end{tabular}
      \endgroup
      \end{center}
  \end{beamerboxesrounded}
  \vspace{\baselineskip}\pause
  \begin{beamerboxesrounded}{Sequencing multiple terms}
    \begin{center}
      \begingroup
      \setlength{\tabcolsep}{2pt}
      \begin{tabular}{rll}
        $\bb{\cdot}_s$ & $\cdot$ &: $Term^* \times (Var^* \rightarrow Anf) \rightarrow Anf$\\
        $\bb{t\ldots}_s$ & $k$ & $= \bb{t\ldots}_s^{\epsilon}$\\
        \\
        $\bb{\epsilon}_s^{x\ldots}$ & $k$ & $= k\,(x\ldots)$\\
        $\bb{t\,t_r\ldots}_s^{x_{acc}\ldots}$ & $k$ & 
        $= \anf{t}{\atomic{\lambda x . \bb{t_r\ldots}_s^{x_{acc}\ldots x}}}$
      \end{tabular}
      \endgroup
      \end{center}
  \end{beamerboxesrounded}
\end{frame}

\begin{frame}[fragile]{After ANF Transformation}
  \begin{lstlisting}
(def eval (env term)
  (match term
    ([String x] (env x))
    ({Abs x body}
      (fun (v)
        (let var1 (extend env x v))
        (eval var1 body)))
    ({App fn arg}
      (let var2 (eval env fn))
      (let var3 (eval env arg))
      (var2 var3))))
  \end{lstlisting}
\end{frame}

\begin{frame}{Control Flow Analysis}
  \begin{itemize}
    \item For each expression in a program, find the over-approximation of~the~set of functions it may evaluate to\pause
    \item Detects function spaces for defunctionalization\pause
    \item Enables selective cps translation
    \pause
    \item Textbook approaches:\pause
    \begin{itemize}
      \item Constraint systems\pause
      \item Annotated type systems
    \end{itemize}
  \end{itemize}
\end{frame}

\begin{frame}{Abstracting Abstract Machines}
  \begin{itemize}
    \item Begin with a CESK* machine extended with patern matching\\
      (Control Environment Store Kontinuation pointer)\pause
    \item Because the program is in ANF:\pause
    \begin{itemize}
      \item There are only two types of continuations\pause
      \item Every proper subexpression will allocate a value in the store
    \end{itemize}\pause
    \item Provide finite approximations for base types\pause
    \item Let the store map to sets of value approximations\pause
    \item Compute the set of configurations reachable from the initial one\pause
    \item Now for every variable in function position the store contains a set of values to which it may evaluate\pause
    \item Methodology introduced by Van Horn et~al. in \textit{Abstracting Abstract Machines}
  \end{itemize}
\end{frame}

\begin{frame}{AAM -- Performance Considerations}
  \begin{itemize}
    \item Use a single shared store (corresponds to widening)\pause
    \item Trim environment in closures and continuations\pause
    \item Sufficiently fast even with naive fixed-point iteration
  \end{itemize}
\end{frame}

\begin{frame}{Selective Translation to CPS}
  \begin{itemize}
    \item Allows the user to specify which functions should be left in direct style \pause
    \item Functions in direct style may call CPS ones and vice versa \pause
    \item Quite simple since the program is already in ANF \pause
    \item Designed not to duplicate code \pause
    \item Produces a program in ANF
  \end{itemize}
\end{frame}

\begin{frame}[fragile]
  \frametitle{CPS Annotations}
  \begin{lstlisting}
(def init #:atomic (x) (error "empty environment"))

(def extend #:atomic (env y v)
  (fun #:atomic (x) (match (eq? x y)
    (#t v)
    (#f (env x)))))
  \end{lstlisting}
\end{frame}

\begin{frame}[fragile]{Selective Translation to CPS}
  \begin{beamerboxesrounded}{Translation to CPS}
    \centering
    \small
  \begin{tabular}{rl}
    $\cps{x}{k}$ &= \lstinline!($k$ $x$)!\\
    
    $\cps{b}{k}$ &= \lstinline!(let $x$ $b$ ($k$ $x$))!\\
    
    $\cps{\lstinline!\{$r\;x\ldots$\}!}{k}$
    &= \lstinline!(let $y$ {$r\;x\ldots$} ($k$ $y$))!\\
  
    $\cps{\lstinline!(fun #:atomic ($x\ldots$)\ $e$)!}{k}$
    &= \lstinline!(let $y$ (fun ($x\ldots$) $\dir{e}$) ($k$ $y$))!\\
  
    $\cps{\lstinline!(fun ($x\ldots$)\ $e$)!}{k}$
    &= \lstinline!(let $y$ (fun ($x\ldots k'$) $\cps{e}{k'}$) ($k$ $y$))!\\
  
    $\cps{\lstinline!($f^l\;x\ldots$)!}{k}$
    &= $ \begin{cases}
      \lstinline!($f$ $x\ldots$ $k$)! \\\quad\mathrm{when}\,\noneAtomic(l)\\
      \lstinline!(let $y$ ($f$ $x\ldots$) ($k$ $y$))! \\\quad\mathrm{when}\,\allAtomic(l)\\
    \end{cases} $\\
  
    $\cps{\lstinline!(match$\;x\;$($p\;e$)$\ldots$)!}{k}$
    &= \lstinline!(match $x$ ($p$ $\cps{e}{k}$)$\ldots$)!\\
  
    $\cps{\lstinline!(let$\;x\;c\;e$)!}{k} $
    &= $ \begin{cases}
      \lstinline!(let $x$ $\dir{c}$ $\cps{e}{k}$)!\\\quad\mathrm{when}\,\trivial(c)\\
      \lstinline!(let $k'$ (fun ($x$) $\cps{e}{k}$) $\cps{c}{k'}$)!\\\quad\mathrm{otherwise}
    \end{cases}$\\
  
    $\cps{\lstinline!(error$\;s$)!}{k}$ &= \lstinline!(error $s$)!
  \end{tabular}
  \end{beamerboxesrounded}
\end{frame}

\begin{frame}[fragile]{Selective Translation to CPS}
  \begin{beamerboxesrounded}{Translation for terms left in direct style}
    \small
    \centering
  \begin{tabular}{rl}
    $\dir{x}$ &= $x$\\
    
    $\dir{b}$ &= $b$\\
    
    $\dir{\lstinline!\{$r\;x\ldots$\}!}$
    &= \lstinline!{$r\;x\ldots$}!\\
  
    $\dir{\lstinline!(fun #:atomic ($x\ldots$)\ $e$)!}$
    &= \lstinline!(fun ($x\ldots$) $\dir{e}$)!\\
  
    $\dir{\lstinline!(fun ($x\ldots$)\ $e$)!}$
    &= \lstinline!(fun ($x\ldots k'$) $\cps{e}{k'}$)!\\
  
    $\dir{\lstinline!($f^l\;x\ldots$)!}$
    &= $ \begin{cases}
      \lstinline!($f$ $x\ldots$)!\\\quad\mathrm{when}\,\allAtomic(l)\\
      \lstinline!(let $k$ (fun ($y$) $y$) ($f$ $x\ldots$ $k$))!\\\quad\mathrm{when}\,\noneAtomic(l)\\
    \end{cases} $\\
  
    $\dir{\lstinline!(match$\;x\;$($p\;e$)$\ldots$)!}$
    &= \lstinline!(match $x$ ($p$ $\dir{e}$)$\ldots$)!\\
  
    $\dir{\lstinline!(let$\;x\;$($f^l\;y\ldots$)$\;e$)!}$
    &= \begin{lstlisting}
(let $k$ (fun ($z$) $z$)
  (let $x$ ($f$ $y\ldots$ $k$))
    $\dir{e}$)
    \end{lstlisting}\quad when $\noneAtomic(l)$\\
  
    $\dir{\lstinline!(let$\;x\;c\;e$)!}$
    &= \lstinline!(let $x$ $\dir{c}$ $\dir{e}$)!\\
  
    $\dir{\lstinline!(error$\;s$)!}$ &= \lstinline!(error $s$)!
  \end{tabular}
  \end{beamerboxesrounded}  
\end{frame}


\begin{frame}[fragile]{After CPS Translation}
  \begin{lstlisting}
(def eval (env term cont)
  (match term
    ([String x]
      (let val (env x))
      (cont val))
    ({Abs x body}
      (let val1
        (fun (v cont1)
          (let var1 (extend env x v))
          (eval var1 body cont1)))
      (cont val1))
    ({App fn arg}
      (eval env fn (fun (var2)
        (eval env arg (fun (var3)
          (var2 var3 cont))))))))
  \end{lstlisting}
\end{frame}

\begin{frame}[fragile]{Selective Defunctionalization}
  \begin{itemize}
    \item Control flow analysis provides all the necessary information\pause
    \item Trade-offs and gotchas \pause
    \begin{itemize}
      \item Direct vs indirect calls to top-level functions\pause
      \item Primitive operations\pause
    \end{itemize}
    \item Record name generation \pause
    \begin{itemize}
      \item For anonymous functions: let the user decide\pause
      \item For continuations: use the constructor of current branch (heuristic)\pause
    \end{itemize}
    \item Result is no longer in ANF
  \end{itemize}
\end{frame}

\begin{frame}[fragile]{Defunctionalization Annotations}
  \begin{lstlisting}
(def init #:no-defun (x) (error "empty environment"))

(def extend (env y v)
  (fun #no-defun (x) (match (eq? x y)
    (#t v)
    (#f (env x)))))

(def eval (env term)
  (match term
    ([String x] (env x))
    ({Abs x body} (fun #:name Fun #:apply apply (v) (eval (extend env x v) body)))
    ({App fn arg} ((eval env fn) (eval env arg)))))
  \end{lstlisting}
\end{frame}

\begin{frame}[fragile]{Selective Defunctionalization}
  \begin{beamerboxesrounded}
    \centering
\begin{tabular}{rl}

  $\bb{x}$ &= $ \begin{cases}
    \lstinline!{Prim$_x$}! & \mathrm{when}\,\mathit{primOp}(x)\\
    % x\text{is a reference to prim op}\\
    \lstinline!{Top$_x$}!  & \mathrm{when}\,\mathit{topLevel}(x) \wedge \defun(x)\\
    % x\text{is a reference to a top-level function which should be defunctionalized}\\
    x & \mathrm{otherwise}
  \end{cases} $\\

  $\bb{b}$ &= $b$\\
  
  $\bb{\lstinline!\{$r\;x\ldots$\}!}$
  &= \lstinline!{r $\bb{x}\ldots$}!\\

  $\bb{\lstinline!(fun\ ($x\ldots$)\ $e$)$^l$!}$
  &= $\begin{cases}
    \lstinline!{Fun$_l$ $\mathit{fvs}(e)$}! &\text{when }\defun(l)\\
    \lstinline!(fun ($x\ldots$) $\bb{e}$)! &\text{otherwise}
  \end{cases}$\\

  $\bb{\lstinline!(!f^{l'}\;x\ldots\lstinline!)!^l}$
  &= $\begin{cases}
    \lstinline!($f\;\bb{x}\ldots$)!&\text{when }\mathit{primOp}(f)\vee\\\quad\mathit{topLevel}(f)\\
    \lstinline!(apply$_l$ $f$ $\bb{x}\ldots$)! &\text{else when }\mathit{allDefun}(l')\\
    \lstinline!($f\;\bb{x}\ldots$)! &\text{when }\mathit{noneDefun}(l')
  \end{cases}$\\

  $\bb{\lstinline!(match$\;x\;$($p\;e$)$\ldots$)!}$
  &= \lstinline!(match $x$ ($p$ $\bb{e}$)$\ldots$)!\\

  $\bb{\lstinline!(let$\;x\;c\;e$)!}$
  &= \lstinline!(let $x$ $\bb{c}$ $\bb{e}$)!\\

  $\bb{\lstinline!(error$\;s$)!}$ &= \lstinline!(error $s$)!
\end{tabular}
  \end{beamerboxesrounded}  
\end{frame}

\begin{frame}[fragile]{Selective Defunctionalization}
  \begin{beamerboxesrounded}{Apply function generation}
    \centering
\begingroup
\setlength{\tabcolsep}{2pt}
\begin{tabular}{rrl}
  $\mkBranch(x\ldots,$&$\delta)$
  &= \lstinline!({Prim$_\delta$} ($\delta\;x\ldots$))!\\

  $\mkBranch(x\ldots,$&$\lstinline!(def $f$ ($y\ldots$) e)!)$
  &= \lstinline!({Top$_f$} ($f$ $x\ldots$))!\\

  $\mkBranch(x\ldots,$&$\lstinline!(fun ($y\ldots$) $e$)$^l$!)$
  &= \lstinline!({Fun$_l$ $\mathit{fvs}(e)$} $\bb{e}[y\mapsto x\ldots]$)!\\
\end{tabular} \pause
\begin{tabular}{rl}
  $\mkApply(l, \mathit{fn}\ldots)$ 
  &= \begin{lstlisting}
(def apply$_l$ ($f$ $x\ldots$)
  (match $f$
    $\mkBranch(x\ldots, \mathit{fn})\ldots$))
  \end{lstlisting}
\end{tabular}
\endgroup
    
  \end{beamerboxesrounded}
\end{frame}

\begin{frame}[fragile]{After Defunctionalization}
  \small
  \begin{lstlisting}
(def eval (env term cont)
  (match term
    ([String x]
      (let val (env x))
      (continue1 cont val))
    ({Abs x body}
      (let val1 {Fun body env x})
      (continue1 cont val1))
    ({App fn arg} (eval env fn {App1 arg cont env}))))
(def apply (fn1 v cont1)
  (match fn1
    ({Fun body env x}
      (let var1 (extend env x v))
      (eval var1 body cont1))))
(def continue1 (fn2 var3)
  (match fn2
    ({App2 cont var2} (apply var2 var3 cont))
    ({App1 arg cont env} (eval env arg {App2 cont var3}))
    ({Halt } var3)))
  \end{lstlisting}
\end{frame}

\begin{frame}{Finishing Touches}
  \begin{itemize}
    \item Inline let-bindings generated by transformation\\
    Only if the variable is used exactly once \pause
    \item Future work: sugar single branch matches as let bindings
  \end{itemize}
\end{frame}

\begin{frame}[fragile]{The Result}
  \begin{lstlisting}
(def eval (env term cont)
  (match term
    ([String x] (continue1 cont (env x)))
    ({Abs x body} (continue1 cont {Fun body env x}))
    ({App fn arg} (eval env fn {App1 arg cont env}))))

(def apply (fn1 v cont1)
  (match fn1 ({Fun body env x} (eval (extend env x v) body cont1))))

(def continue1 (fn2 var3)
  (match fn2
    ({App2 cont var2} (apply var2 var3 cont))
    ({App1 arg cont env} (eval env arg {App2 cont var3}))
    ({Halt } var3)))
  \end{lstlisting}
\end{frame}

\begin{frame}
  \centering
  \LARGE{Demo: Actually Using the Tool}\\
  \small{call-by-value normalization by evaluation}
\end{frame}

% \begin{frame}{Demo}
%   \centering
%   An evaluator for call-by-name \LC{}
% \end{frame}

\section{Conclusion}
\begin{frame}{Case studies}
  \begin{center}
  \scriptsize
  \begin{tabular}{c|c|c|c}
  Language & Interpreter style & Lang. Features & Result \\ 
  \Xhline{2\arrayrulewidth}
  \multirow{13}{*}{\makecell{call-by-value \\ \LC{}}} & denotational & $\cdot$ & CEK machine \\
  \cline{2-4}
  & denotational & integers with add & CEK with add \\
  \cline{2-4}
  & \makecell{denotational, \\ recursion via \\ environment} & \makecell{integers, recursive \\ let-bindings} & \makecell{similar to Reynold's  \\ first-order interpreter}\\
  \cline{2-4}
  & \makecell{denotational \\ with conts.} & shift and reset & two layers of conts.\\
  \cline{2-4}
  & \makecell{denotational, \\ monadic} & \multirow{3}{*}{\makecell{exceptions \\ with handlers}} & \makecell{explicit \\ stack unwinding}\\
  \cline{2-2}\cline{4-4}
  & \makecell{denotational, \\ CPS} & & \makecell{pointer to\\ exception handler}\\
  \cline{2-4}
  & \makecell{normalization \\ by evaluation} & $\cdot$ & strong CEK machine \\
  \hline
  \makecell{call-by-name \\ \LC{}} & big-step & $\cdot$ & Krivine machine \\
  \hline
  \makecell{call-by-need \\ \LC{}} & \makecell{big-step \\ (state passing)} & memoization & lazy Krivine machine \\
  \hline
  \makecell{simple \\ imperative} & \makecell{big-step \\ (state passing)} & \makecell{conditionals, \\ while, assignment} & $\cdot$\\
  \hline
  micro-Prolog & CPS & \makecell{backtracking, \\ cut operator} & logic engine\\
  \hline
  \end{tabular}
  \end{center}
\end{frame}

\begin{frame}{Conclusion}
  \begin{itemize}
    \item Algorithm \pause
    \begin{itemize}
      \item Fully automatic transformation \pause
      \item Works with interpreters expressed in a higher-order language \pause
      \item Allows for fine-grained control over the resulting machine \pause
    \end{itemize}
    \pause
    \item Implementation \pause
    \begin{itemize}
      \item Interpreters embedded in \textit{Racket} source files \pause
      \item Users influence transformation using annotations \pause
      \item Tested on a selection of interpreters
    \end{itemize}
    \pause
    \item Current work: \textit{Coq} formalization \pause
    \begin{itemize}
      \item First version: encodings of natural semantics \pause
      \item Define transformation on a deeply embedded language \pause
      \item Leverage \textit{MetaCoq} library to transform to and from shallow embeddings
    \end{itemize}
  \end{itemize}  
\end{frame}

\begin{frame}
  \begin{itemize}
    \item Reynolds: \textit{Definitional Interpreters for Higher-Order Programming Languages}
    \item Ager, Biernacki, Danvy and Midtgaard: \textit{A functional correspondence between evaluators and abstract machines}
    \item Van Horn and Might: \textit{Abstracting abstract machines}
  \end{itemize}
\end{frame}

% \begin{frame}{Case studies}
%   \begin{center}
%   \scriptsize
%   \begin{tabular}{c|c|c|c}
%   Language & Interpreter style & Lang. Features & Result \\ 
%   \Xhline{2\arrayrulewidth}
%   \multirow{13}{*}{\makecell{call-by-value \\ \LC{}}} & denotational & $\cdot$ & CEK machine \\
%   \cline{2-4}
%   & denotational & integers with add & CEK with add \\
%   \cline{2-4}
%   & \makecell{denotational, \\ recursion via \\ environment} & \makecell{integers, recursive \\ let-bindings} & \makecell{similar to Reynold's  \\ first-order interpreter}\\
%   \cline{2-4}
%   & \makecell{denotational \\ with conts.} & shift and reset & two layers of conts.\\
%   \cline{2-4}
%   & \makecell{denotational, \\ monadic} & \multirow{3}{*}{\makecell{exceptions \\ with handlers}} & \makecell{explicit \\ stack unwinding}\\
%   \cline{2-2}\cline{4-4}
%   & \makecell{denotational, \\ CPS} & & \makecell{pointer to\\ exception handler}\\
%   \cline{2-4}
%   & \makecell{normalization \\ by evaluation} & $\cdot$ & strong CEK machine \\
%   \hline
%   \makecell{call-by-name \\ \LC{}} & big-step & $\cdot$ & Krivine machine \\
%   \hline
%   \makecell{call-by-need \\ \LC{}} & \makecell{big-step \\ (state passing)} & memoization & lazy Krivine machine \\
%   \hline
%   \makecell{simple \\ imperative} & \makecell{big-step \\ (state passing)} & \makecell{conditionals, \\ while, assignment} & $\cdot$\\
%   \hline
%   micro-Prolog & CPS & \makecell{backtracking, \\ cut operator} & logic engine\\
%   \hline
%   \end{tabular}
%   \end{center}
% \end{frame}

\end{document}
