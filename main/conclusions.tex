\chapter{Conclusions}\label{chapter:conclusions}
In the thesis I described an algorithm allowing for automatic derivation of an abstract machine given an interpreter which usually corresponds to denotational or natural semantics.
The algorithm allows for specification of functions which should be considered atomic from the point of view of control-flow and function spaces which treated as abstract (i.e., left in the higher-order form) in the resulting machine.
In order to enable the transformation I derived the control-flow analysis for \IDL{} using the abstracting abstract machines methodology.
I implemented the algorithm in the \textit{Haskell} programming language and used this tool to transform a selection of interpreters.

The correctness of the tool has been established experimentally by running the interpreters after every intermediate step of transformation.
The next logical step is a formalization of the algorithm using a proof assistant (e.g., Coq) to obtain a powerful and correct method of deriving abstract machines.

In order to extend capabilities of \semt{} as a practical tool for semantics engineering the future work could include extending the set of primitive operations and adding the ability to import arbitrary \textit{Racket} functions and provide their abstract specification.
Another important matter is the performance (i.e., speed) of the tool.
To this end a thorough investigation of cost and complexity of computing the control-flow analysis is required.

Other avenue for improvement lies in extensions of the meta-language capabilities.
Investigation of additions such as control operators, nondeterministic choice and concurrency could yield many opportunities for diversifying the set of interpreters (and languages) that may be encoded in the \IDL{}.
In particular control operators could allow for expressing the interpreter for a language with delimited control (or algebraic effects) in direct style.