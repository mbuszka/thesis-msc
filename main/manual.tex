\chapter{User's Manual}\label{chapter:user-manual}
\subsection*{Installation}
The semantic transformer -- \semt{} is built from source using \lstinline!cabal! -- a \textit{Haskell} package manager.
To build the binary use \lstinline!cabal build! and to install the resulting binary use \lstinline!cabal install! in the root of the project.

\subsection*{Tool usage}
The basic mode of usage is to transform a \lstinline!file.rkt! containing an interpreter into \lstinline!out/file.rkt! which is a source file containing the transformed interpreter, i.e., an abstract machine using the command \lstinline!semt file.rkt!.
The options modifying the behavior of the tool can be displayed with the command \lstinline!semt --help!:
\begin{lstlisting}
Usage: semt FILE [-o|--output DIR] [-i|--intermediate] 
            [-d|--debug] [-t|--self-test]
  Transform an interpreter into an abstract machine.

Available options:
  FILE                     Source file with the interpreter.
  -o,--output DIR          Output directory for generated
                            files, defaults to ./out/
  -i,--intermediate        Emit executable source files for
                            each stage.
  -d,--debug               Emit labeled source files for
                            each stage.
  -t,--self-test           Run raco test on each
                            intermediate file; implies
                            --intermediate
  -h,--help                Show this help text
\end{lstlisting}

\begin{figure}
  \lstinputlisting[numbers=left]{interpreters/src/lambda-value.rkt}
  \caption{An example of a source file}
  \label{fig:idl-example}
\end{figure}

\subsection*{Source File Format}
The tool assumes that the source file with an interpreter is a \textit{Racket} program.
An example source listing is shown in Figure \ref{fig:idl-example}.
The preamble (everything up to the \lstinline!; begin interpreter! marker in line 4) is copied verbatim by the tool and should be used to specify \emph{Racket}'s dialect (line 1) and import syntax definitions (line 2).
Afterwards an interpreter is specified (lines 4 -- 31) and it ends with another marker \lstinline!; end interpreter! in line 32.
Everything following the marker is again copied to the output file.
In the example this space is used to define some tests for the interpreter.
The syntax of \IDL{} is given in Figure \ref{fig:idl-syntax}.

\subsection*{Top-level definitions}
The interpreter consists of a sequence of datatype (\textit{data-def}), record (\textit{struct-def}) and function (\textit{fun-def}) definitions.
One of the functions must be named \lstinline!main! and will serve as an entry point of the interpreter.
It is required for the \lstinline!main! function to have parameters annotated with types, e.g., \lstinline!(def main ([String foo]$\;$[Any e])$\;$body)!.
Names (\textit{tp-name}) of datatypes and records must be unique distinct from base types (\textit{base-tp} and \lstinline!Any!).
All definitions may be mutually recursive.

\subsection*{Terms}
The term syntax is split into terms and statements where the latter appear as the bodies of function definitions (both top-level and anonymous) and branches.
Statements are sequences of let-bindings terminated by a regular term (see lines 16--19 of Figure \ref{fig:idl-example}).

\subsection*{Execution}
In order to run the interpreter a library with syntax definitions must be imported.
Let us assume the following directory structure:
\begin{lstlisting}
interpreters/
  src/
    lambda-value.rkt
  lib/
    idl.rkt
\end{lstlisting}
Where \lstinline!idl.rkt! is the library of syntax definitions provided with this thesis and \lstinline!lambda-value.rkt! is the file we want to execute.
Then it suffices to add \lstinline!(require "../lib/idl.rkt")! (line 2 of the example) at the top of the file.

The run-time behavior of the program follows the call-by-value behavior of \emph{Racket}, e.g.:
\begin{itemize}
  \item \lstinline!let! evaluates the bound term to a value before binding.
  The binding is visible only in following statements.
  \item Application evaluates terms left-to-right and then applies the value in function position.
  \item \lstinline!match! evaluates the term under scrutiny and tests patterns against the value in order of their definition continuing with the first match.
\end{itemize}

\subsection*{Annotations}
\begin{itemize}
  \item \lstinline!#:apply! specifies the name for \emph{apply} function generated for the defunctionalized function space whose member is the annotated function.
  \item \lstinline!#:name! specifies the name for the record which will represent the annotated function after defunctionalization.
  \item \lstinline!#:no-defun! skips defunctionalization of the annotated function (either all or none of the functions in a space must be annotated).
  \item \lstinline!#:atomic! means that the annotated function (and calls to it) will be left in direct style during translation to CPS.
\end{itemize}




\begin{figure}
\centering
\begin{tabular}{rl}
$\mathit{data\mh def}$
& ::= \lstinline!(def-data $\mathit{tp\mh name}$ $\mathit{tp\mh elem}\dots$)!\\

$\mathit{tp\mh elem}$
& ::= $\mathit{tp}$ | $\mathit{record}$\\

$\mathit{record}$
& ::= \lstinline!{$\mathit{tp\mh name}$ $\mathit{record\mh field}\ldots$}!\\

$\mathit{record\mh field}$
& ::= $\mathit{tp}$
  | $\mathit{var}$
  | \lstinline![$\mathit{tp}$ $\mathit{var}$]!\\

$\mathit{record\mh def}$
& ::= \lstinline!(def-struct $\mathit{record}$)!\\

$\mathit{base\mh tp}$
& ::= \lstinline!String! | \lstinline!Integer! | \lstinline!Boolean!\\

$\mathit{tp}$
& ::= \lstinline!Any! | $\mathit{base\mh tp}$ | $\mathit{tp\mh name}$\\

$\mathit{fun\mh def}$
& ::= \lstinline!(def $\mathit{var}$ $\mathit{annot}\ldots$ ($\mathit{arg}\ldots$) $\mathit{statements}$)!\\

$\mathit{annot}$
& ::= \lstinline!#:no-defun!
    | \lstinline!#:atomic!
    | \lstinline!#:name !$\mathit{tp\mh name}$
    | \lstinline!#:apply !$\mathit{var}$\\

$\mathit{arg}$
& ::= $\mathit{var}$ | \lstinline![$\mathit{tp}$ $\mathit{var}$]!\\

$\mathit{const}$
& ::= $\mathit{integer}$ | $\mathit{string}$ | \lstinline!#t! | \lstinline!#f!\\

$\mathit{statements}$
& ::= \lstinline!(let $\mathit{var}$ $\mathit{term}$)! $\mathit{statements}$
    | $\mathit{term}$\\

$\mathit{term}$
& ::= $\mathit{var}$ 
    | \lstinline!(fun $\mathit{annot}\ldots$ ($\mathit{arg}\ldots$) $\mathit{statements}$)!
    | \lstinline!($\mathit{term}$ $\mathit{term}\ldots$)!\\
&   | \lstinline!{$\mathit{tp\mh name}$ $\mathit{term}\ldots$}!
    | \lstinline!(match $\mathit{term}$ $\mathit{branch}\ldots$)!
    | \lstinline!(error $\mathit{string}$)!\\

$\mathit{branch}$
& ::= \lstinline!($\mathit{pattern}$ $\mathit{statements}$)!\\

$\mathit{pattern}$
& ::= $\mathit{var}$ | $\mathit{const}$ | \lstinline!_! | \lstinline![$\mathit{base\mh tp}$ $\mathit{var}$]! | \lstinline!{$\mathit{tp\mh name}$ $\mathit{pattern}\ldots$}!\\
\end{tabular}
\caption{Syntax of \IDL{}}
\label{fig:idl-syntax}
\end{figure}