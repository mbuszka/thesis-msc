\documentclass{beamer}

\mode<presentation>
{
  \usetheme{Boadilla}      % or try Darmstadt, Madrid, Warsaw, ...
  \usecolortheme{default} % or try albatross, beaver, crane, ...
  \usefonttheme{default}  % or try serif, structurebold, ...
  \setbeamertemplate{navigation symbols}{}
  \setbeamertemplate{caption}[numbered]
} 
\usepackage[polish]{babel}
\usepackage{amsmath}
\usepackage{amssymb}
\usepackage[T1]{fontenc}
\usepackage{multirow}
\usepackage{makecell}
\usepackage{amsfonts}
\usepackage{stmaryrd}


\usepackage{xcolor}
\usepackage{listings}
\usepackage[scale=0.8]{FiraMono} 

\lstdefinelanguage{idl}{%
  alsoletter={-?\#:},
  otherkeywords={(,),\{,\},[,]},
  morekeywords={def,fun,def-data,def-struct,match,let,error},
  morekeywords=[2]{String,Integer,Any,Boolean},
  morekeywords=[3]{(,),\{,\},[,]},
  morekeywords=[4]{\#:bar,\#:apply,\#:atomic,\#:no-defun,\#:name,\#t, \#f, \#:foo},
  sensitive=true,
  morestring=[b]",
  morecomment=[l]{;;}
}

\lstset{%
  basicstyle=\ttfamily\color{black},
  mathescape=true,
  showstringspaces=false,
  keywordstyle={\color{green!50!black}},
  keywordstyle=[2]{\color{blue!50!black}}, 
  keywordstyle=[3]{\color{gray}},
  keywordstyle=[4]{\color{blue!80!black}},
  commentstyle=\color{gray}\itshape,
  stringstyle=\color{green!80!black},
  identifierstyle=\color{black},
  escapechar=@,
  language=idl
} 

\newcommand{\Redex}{\texttt{PLT Redex}}
\newcommand{\Racket}{\texttt{Racket}}
\newcommand{\LC}{\(\lambda\)-calculus}

\title[The Functional Correspondence Applied]{The Functional Correspondence Applied:\\ An Implementation of a Semantics Transformer}
\author[Maciej Buszka]{Maciej Buszka}
\institute[II UWr]{Instytut Informatyki UWr\\{\vspace{\baselineskip}Supervised by: dr. hab. Dariusz Biernacki}}
\date{23.07.2020}

\begin{document}

\begin{frame}
  \titlepage
\end{frame}

\begin{frame}{Outline}
  \tableofcontents
\end{frame}

\section{Introduction}
\begin{frame}{Problem statement}
  \begin{itemize}
    \item Abstract Machines
    \begin{itemize}
      \item Precisely describe operational properties of a program
      % \pause
      \item May serve as an implementation basis
      % \pause
      \item Hard to create from scratch
    \end{itemize}
    \pause
    \item High-level semantics
    % \pause
    \begin{itemize}
      \item Denotational and big-step operational
      % \pause
      \item Concise and abstract
      % \pause
      \item May be understood intuitively
    \end{itemize}
    \pause
    \item Deriving abstract machines
    \begin{itemize}
      % \pause
      \item Requires proof of correctness
      % \pause
      \item Mechanized by the functional correspondence
    \end{itemize}
  \end{itemize}
\end{frame}

\begin{frame}{Goals}
  \begin{itemize}
    \item An algorithm deriving an abstract machine
    \begin{itemize}
      \item Takes an interpreter in a functional language
      \item Produces an encoding of an abstract machine
      \item Fully automatic
    \end{itemize}
    \pause
    \item A practical tool which implements this algorithm
    \begin{itemize}
      \item Gives control over the shape of the result
      \item Generates readable machines
      \item Allows for testing the interpreters
    \end{itemize}
  \end{itemize}
\end{frame}

\section{The Functional Correspondence}

\begin{frame}{The Functional Correspondence}
  \begin{itemize}
    \item A manual method of deriving abstract machines
    \item Starts with an evaluator
    \item Finishes with an encoding of an abstract machine
    \pause
    \item Based on two source-to-source transformations
    \begin{itemize}
      \item Translation to continuation-passing style (CPS)
      \item Defunctionalization
    \end{itemize}
    \pause
    \item Successfully applied to a multitude of diverse evaluators
  \end{itemize}
\end{frame}

\begin{frame}[fragile]
\frametitle{Running example: call-by-name \LC{}}
\begin{lstlisting}
(def-data Term
  Integer         ;; de Bruijn index
  {App Term Term} ;; application
  {Abs Term})     ;; abstraction

(def eval (expr env)
  (match expr
    ([Integer n] ((env n)))
    ({App f x}
      ((eval f env) (fun () (eval x env))))
    ({Abs body}
      (fun (x) (eval body (cons x env))))))
\end{lstlisting}
\end{frame}


\subsection{Translation to CPS}

\begin{frame}{Translation to CPS}
  \begin{itemize}
    \item Goal: expose control-flow of an interpreter
    \item Classify functions into trivial and serious ones
    \begin{itemize}
      \item Serious functions may only be called in tail position
      \item Trivial functions may be called anywhere
    \end{itemize}
    \pause
    \item Pass additional argument -- the continuation
    \begin{itemize}
      \item Specifies what should be done after the function finishes
      \item Allows to express interesting programs while retaining tail-call property
    \end{itemize}
  \end{itemize}
\end{frame}

\begin{frame}[fragile]
  \frametitle{Interpreter in CPS}
  \begin{lstlisting}
(def eval (expr env cont)
  (match expr
    ([Integer n] ((env n) cont))
    ({App f x}
      (eval f env
        (fun (var3)
          (var3 (fun (cont1) (eval x env cont1)) cont))))
    ({Abs body}
      (cont (fun (x cont2) (eval body (cons x env) cont2))))))
  \end{lstlisting}
\end{frame}

\subsection{Defunctionalization}

\begin{frame}{Defunctionalization}
  \begin{itemize}
    \item Goal: produce first-order program
    \item For each function space
    \begin{itemize}
      \item Transform anonymous function definitions into records holding the free variables
      \item Generate top-level function which matches the records and evaluates the bodies
      \item Transform applications of functions in the space into a call to the top-level function
    \end{itemize}
  \end{itemize}
\end{frame}

\begin{frame}[fragile]
  \frametitle{Resulting Machine}
  \begin{lstlisting}
(def eval (expr env cont)
  (match expr
    ([Integer n] (force (env n) cont))
    ({App f x} (eval f env {App1 cont env x}))
    ({Abs body} (continue cont {Clo body env}))))

(def force (fn cont1)
  (match fn ({Thunk env x} (eval x env cont1))))

(def apply (fn1 x cont2)
  (match fn1 ({Clo body env} (eval body (cons x env) cont2))))

(def continue (fn2 var3)
  (match fn2
    ({App1 cont env x} (apply var3 {Thunk env x} cont))
    ({Halt} var3)))
  \end{lstlisting}
\end{frame}

\section{The Semantics Transformer}
% \begin{frame}
%   \centering
%   \huge The Semantics Transformer
% \end{frame}
% \begin{frame}{Practical Considerations}
%   \begin{itemize}
%     \item Interpreters embedded in \textit{Racket} source files
%     \begin{itemize}
%       \item Natural, functional meta-language with pattern matching
%       \item May be executed in REPL, loaded from other modules etc.
%       \item Can be tested using \textit{Racket}'s framework
%       \item Tests may use full power of the language and macros
%     \end{itemize}
%     \item Machine generated using only the source file
%     \begin{itemize}
%       \item No need to write additional configuration, flags, etc.
%       \item Shape of the result may be modified by annotating source program
%     \end{itemize}
%     \item The tool can pretty print intermediate results
%     \item Names are generated mostly deterministically
%   \end{itemize}
% \end{frame}

\subsection{Control-flow Analysis}

\begin{frame}{Control-flow Analysis}
  \begin{itemize}
    \item For each expression in a program, find the over-approximation of the set of functions it may evaluate to
    \item Exactly matches requirements of defunctionalization
    \pause
    \item Textbook approaches
    \begin{itemize}
      \item Constraint systems
      \item Annotated type systems
      \item Subjectively hard to adapt to the meta-language
    \end{itemize}
  \end{itemize}
\end{frame}

\begin{frame}{Abstracting Abstract Machines}
  \begin{itemize}
    \item Derive an analysis from abstract machine
    \begin{itemize}
      \item Mechanical, principled process
      \item Easy to adapt various language features
    \end{itemize}
    \item Results of analysis fit the functional correspondence well
    \item Reasonable running time on small (100 lines) interpreters even with very naive implementation
  \end{itemize}
\end{frame}


% \subsection{A-normal Form}

% \begin{frame}{Translation to A-normal Form}
%   \begin{itemize}
%     \item Intermediate representation of programs
%     \item Every expression has only variables as subterms
%     \item Works by let-binding intermediate results
%     \item Simplifies analysis and subsequent transformations
%   \end{itemize}
% \end{frame}


\subsection{Selective Translation to CPS}

\begin{frame}{Selective Translation to CPS}
  \begin{itemize}
    \item Extension of standard CPS translation
    \begin{itemize}
      \item Allow to specify which functions should be left in direct style
      \item Functions in direct style may call CPS ones and vice versa
    \end{itemize}
    \item Uses control-flow analysis to guide transformation of applications
    \item Beneficial in practice -- machine is not cluttered with control flow of helper functions
  \end{itemize}
\end{frame}

\begin{frame}[fragile]
  \frametitle{CPS Annotations}
  \begin{lstlisting}
(def cons #:atomic (val env)
  (fun #:atomic (n)
    (match n
      (0 val)
      (_ (env (- n 1))))))

(eval term (fun #:atomic (n) (error "empty env")))
  \end{lstlisting}
\end{frame}


\subsection{Selective Defunctionalization}

\begin{frame}{Selective Defunctionalization}
  \begin{itemize}
    \item Extends defunctionalization with option to leave selected function spaces untouched
    \item Uses control-flow analysis
    \begin{itemize}
      \item Generation of apply functions
      \item Guide transformation of applications
      \item Pass references to top-level functions as records where necessary
    \end{itemize}
    \item Beneficial in practice -- pieces of machine may be left abstract
  \end{itemize}
\end{frame}

\begin{frame}[fragile]
  \frametitle{Defunctionalization Annotations}
  \begin{lstlisting}
(def cons #:atomic (val env)
  (fun #:atomic #:no-defun (n)
    (match n
      (0 val)
      (_ (env (- n 1))))))

(eval term (fun #:atomic #:no-defun (n) (error "empty env")))
  \end{lstlisting}
\end{frame}

% \begin{frame}{Demo}
%   \centering
%   An evaluator for call-by-name \LC{}
% \end{frame}

\section{Conclusion}
\begin{frame}{Case studies}
  \begin{center}
  \scriptsize
  \begin{tabular}{c|c|c|c}
  Language & Interpreter style & Lang. Features & Result \\ 
  \Xhline{2\arrayrulewidth}
  \multirow{13}{*}{\makecell{call-by-value \\ \LC{}}} & denotational & $\cdot$ & CEK machine \\
  \cline{2-4}
  & denotational & integers with add & CEK with add \\
  \cline{2-4}
  & \makecell{denotational, \\ recursion via \\ environment} & \makecell{integers, recursive \\ let-bindings} & \makecell{similar to Reynold's  \\ first-order interpreter}\\
  \cline{2-4}
  & \makecell{denotational \\ with conts.} & shift and reset & two layers of conts.\\
  \cline{2-4}
  & \makecell{denotational, \\ monadic} & \multirow{3}{*}{\makecell{exceptions \\ with handlers}} & \makecell{explicit \\ stack unwinding}\\
  \cline{2-2}\cline{4-4}
  & \makecell{denotational, \\ CPS} & & \makecell{pointer to\\ exception handler}\\
  \cline{2-4}
  & \makecell{normalization \\ by evaluation} & $\cdot$ & strong CEK machine \\
  \hline
  \makecell{call-by-name \\ \LC{}} & big-step & $\cdot$ & Krivine machine \\
  \hline
  \makecell{call-by-need \\ \LC{}} & \makecell{big-step \\ (state passing)} & memoization & lazy Krivine machine \\
  \hline
  \makecell{simple \\ imperative} & \makecell{big-step \\ (state passing)} & \makecell{conditionals, \\ while, assignment} & $\cdot$\\
  \hline
  micro-Prolog & CPS & \makecell{backtracking, \\ cut operator} & logic engine\\
  \hline
  \end{tabular}
  \end{center}
\end{frame}

\begin{frame}{Conclusion}
  \begin{itemize}
    \item Algorithm
    \begin{itemize}
      \item Fully automatic transformation
      \item Works with interpreters expressed in a higher-order language
      \item Allows for fine-grained control over the resulting machine
    \end{itemize}
    \pause
    \item Implementation
    \begin{itemize}
      \item Interpreters embedded in \textit{Racket} source files
      \item Modification of transformation via annotations
      \item Tested on a selection of interpreters
    \end{itemize}
    \pause
    \item Further work
    \begin{itemize}
      \item Formalization in \textit{Coq}
      \item Transformation of other encodings of semantic formats
      \item Different backends: \textit{C}, \LaTeX
      \item Nondeterministic languages
    \end{itemize}
  \end{itemize}  
\end{frame}

\begin{frame}
  \centering
  \huge Thank You
\end{frame}

\begin{frame}{Case studies}
  \begin{center}
  \scriptsize
  \begin{tabular}{c|c|c|c}
  Language & Interpreter style & Lang. Features & Result \\ 
  \Xhline{2\arrayrulewidth}
  \multirow{13}{*}{\makecell{call-by-value \\ \LC{}}} & denotational & $\cdot$ & CEK machine \\
  \cline{2-4}
  & denotational & integers with add & CEK with add \\
  \cline{2-4}
  & \makecell{denotational, \\ recursion via \\ environment} & \makecell{integers, recursive \\ let-bindings} & \makecell{similar to Reynold's  \\ first-order interpreter}\\
  \cline{2-4}
  & \makecell{denotational \\ with conts.} & shift and reset & two layers of conts.\\
  \cline{2-4}
  & \makecell{denotational, \\ monadic} & \multirow{3}{*}{\makecell{exceptions \\ with handlers}} & \makecell{explicit \\ stack unwinding}\\
  \cline{2-2}\cline{4-4}
  & \makecell{denotational, \\ CPS} & & \makecell{pointer to\\ exception handler}\\
  \cline{2-4}
  & \makecell{normalization \\ by evaluation} & $\cdot$ & strong CEK machine \\
  \hline
  \makecell{call-by-name \\ \LC{}} & big-step & $\cdot$ & Krivine machine \\
  \hline
  \makecell{call-by-need \\ \LC{}} & \makecell{big-step \\ (state passing)} & memoization & lazy Krivine machine \\
  \hline
  \makecell{simple \\ imperative} & \makecell{big-step \\ (state passing)} & \makecell{conditionals, \\ while, assignment} & $\cdot$\\
  \hline
  micro-Prolog & CPS & \makecell{backtracking, \\ cut operator} & logic engine\\
  \hline
  \end{tabular}
  \end{center}
\end{frame}

\end{document}
